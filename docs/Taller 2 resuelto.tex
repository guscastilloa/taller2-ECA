\documentclass[a4paper, answers, addpoints, 11pt]{exam}
%\documentclass{article}
\addtolength{\hoffset}{-1.25cm}
\addtolength{\textwidth}{2.75cm}
\addtolength{\voffset}{-2.0cm}
\addtolength{\textheight}{3cm}
\setlength{\parskip}{0pt}
\setlength{\parindent}{0in}


%----------------------------------------------------------------------------------------
%	PACKAGES AND OTHER DOCUMENT CONFIGURATIONS
%----------------------------------------------------------------------------------------
\usepackage{amsmath}   % For mathematical formatting and equations
\usepackage{amsthm, amsmath, amssymb} % Mathematical typesetting
\newtheorem{definition}{Definición}
\newtheorem{theorem}{Teorema}
\newtheorem{lemma}{Lema}
\usepackage{tikz}
\usetikzlibrary{positioning, arrows.meta}}
\usepackage{algorithm}
\usepackage{algorithmic}
% \usepackage{algpseudocode}

\usepackage{xcolor}
%\usepackage de{quiz}
\usepackage{threeparttable}
\usepackage{framed}
\usepackage{xcolor}
\usepackage{hyperref}  % For hyperlinks and clickable references
\usepackage{enumitem}  % For customizing lists
\usepackage{subcaption}
\usepackage{mdframed}
\usepackage{amsmath}
\usepackage{xcolor}
\usepackage{tcolorbox}  % Para crear la celda sombreada

% Definir el entorno para el cuadro de la solución con color azul

% Define a custom style that mimics the tcolorbox settings:
% \mdfdefinestyle{solutionstyle}{%
%   %backgroundcolor=gray!5,                % light gray background
%   linecolor=green!70!black!50,           % frame color
%   %linewidth=1pt,                         % thickness of the frame lines
%   %roundcorner=0pt,                       % no rounded corners (arc=0mm)
%   frametitle={\bfseries Solución},        % title of the box
%   %frametitlerule=true,                   % draw a rule below the title
%   %frametitlerulewidth=0.4pt,             % thickness of the title rule
%   %frametitlebackgroundcolor=gray!20,     % background behind the title
%   %frametitlealignment=\raggedright,      % title alignment
%   %innerleftmargin=10pt,                  % inner margins
%   %innerrightmargin=10pt,                 % inner right margin
%   %innertopmargin=5pt,                   % inner top margin
%   %innerbottommargin=5pt,                 % inner bottom margin
%   %skipabove=\baselineskip,               % vertical space before and after the box
%   %skipbelow=\baselineskip,
%   %breakable=true,                        % allow page breaks
%   %width=\textwidth,                      % make the box as wide as the text width
%   %rightskip=0pt plus 1fil,               % prevent overflow on the right side
%   %leftskip=0pt plus 1fil,                % prevent overflow on the left side
%   %align=left                             % ensure alignment stays left
% }



% Configuración del entorno de solución

%\newenvironment{solucion}
  %{\begin{solucion}[style=solutionstyle]}
  %{\end{solucion}}
\usepackage{blindtext} % Package to generate dummy text
\usepackage{charter} % Use the Charter font
\usepackage[utf8]{inputenc} % Use UTF-8 encoding
\usepackage{microtype} % Slightly tweak font spacing for aesthetics
\usepackage[english, spanish, es-nodecimaldot]{babel} % Language hyphenation and typographical rules
\usepackage{float} % Improved interface for floating objects
\usepackage[final, colorlinks = true, 
            linkcolor = black, 
            citecolor = black]{hyperref} % For hyperlinks in the PDF
\usepackage{fancyhdr}
\pagestyle{fancy}
\renewcommand{\lhead}{Mi Encabezado}

\usepackage{graphicx, multicol} % Enhanced support for graphics
\usepackage{xcolor} % Driver-independent color extensions
\usepackage{marvosym, wasysym} % 3More symbols
\usepackage{rotating} % Rotation tools
\usepackage{censor} % Facilities for controlling restricted text
%\usepackage{pseudocode} % Environment for specifying algorithms in a natural way
\usepackage{booktabs} % Enhances quality of tables
\usepackage{tikz-qtree} % Easy tree drawing tool
\tikzset{every tree node/.style={align=center,anchor=north},
         level distance=1cm} % Configuration for q-trees
\usepackage[backend=biber,style=numeric,
            sorting=nyt]{biblatex} % Complete reimplementation of bibliographic facilities
\addbibresource{ecl.bib}
\usepackage{csquotes} % Context sensitive quotation facilities
\usepackage[yyyymmdd]{datetime} % Uses YEAR-MONTH-DAY format for dates
\renewcommand{\dateseparator}{-} % Sets dateseparator to '-'
\usepackage{fancyhdr} % Headers and footers
\pagestyle{fancy} % All pages have headers and footers
\fancyhead{}\renewcommand{\headrulewidth}{0pt} % Blank out the default header
\fancyfoot[L]{} % Custom footer text
\fancyfoot[C]{} % Custom footer text
\fancyfoot[R]{\thepage} % Custom footer text
\newcommand{\note}[1]{\marginpar{\scriptsize \textcolor{red}{#1}}} % Enables comments in red on margin
\DeclareMathOperator*{\plim}{plim}
\usepackage[most]{tcolorbox}
\usepackage{adjustbox}
\addto\captionsspanish{
\def\listtablename{\'Indice de tablas}%
\def\tablename{Tabla}}
\usepackage{xcolor}
\usepackage{multirow}
\usepackage{listings}
\usepackage{bbm}
\definecolor{myblue}{RGB}{0,163,243}
\definecolor{moradito}{RGB}{63,1,143}
\definecolor{moraditoClaro}{RGB}{130,130,200}
\definecolor{lightpink}{rgb}{1.0, 0.75, 0.8}

\newenvironment{solucion}{%
  \begin{mdframed}[
    backgroundcolor=blue!5,    % Fondo azul claro para el cuadro
    linecolor=blue!50,          % Borde azul más oscuro
    linewidth=2pt,              % Grosor del borde
    leftmargin=10pt,            % Margen izquierdo
    rightmargin=8pt,           % Margen derecho
    topline=true,              % Sin línea superior
    bottomline=true,            % Línea inferior activada
    roundcorner=10pt,           % Esquinas redondeadas
    innerleftmargin=10pt,       % Margen interno a la izquierda
    innerrightmargin=10pt,      % Margen interno a la derecha
    innerbottommargin=10pt,     % Margen inferior
    innertopmargin=10pt         % Margen superior
  ]%
  \begin{tcolorbox}[colframe=blue!50!black, colback=blue!50, coltitle=white, sharp corners=all, boxrule=1mm, width=\textwidth, halign=left, valign=center, top=0mm, bottom=0mm, left=0mm, right=0mm] \textbf{Solución} \end{tcolorbox} }{\end{mdframed}}
\newcommand{\E}{\mathrm{E}}
\newcommand{\Var}{\mathrm{Var}}
\newcommand{\sVar}{\widehat{\mathrm{var}}}
\newcommand{\Cov}{\mathrm{cov}}
\newcommand{\sCov}{\widehat{\mathrm{cov}}}
\begin{document}

%-------------------------------
%	TITULO
%-------------------------------

\fancyhead[C]{}
\hrule \medskip 
\begin{minipage}{0.295\textwidth} 
\raggedright
\textbf{Profesor:} Manuel Fernández\\
\vspace{2mm}
Lucía Maldonado \\
Juan Felipe Mora \\
Danilo Aristizabal \\
Edmundo Arias De Abreu



\end{minipage}
\begin{minipage}{0.4\textwidth} 
\centering 
\huge 
Taller 2\\ 
\vspace{2mm}
\normalsize 
Econometría Avanzada, 2025-1\\ 
\vspace{2mm}
Presentado por: \\Gustavo Alvarez Castillo- 201812166\\
Ana María Patrón Piñerez -201714291

\end{minipage}

\medskip\hrule 
\bigskip

\section*{Primer Ejercicio}

El Diseño de Regresión Discontinua (RDD o simplemente RD, por sus siglas en inglés) es uno de los métodos no experimentales más utilizados para la inferencia causal y la evaluación de programas. En la actualidad, cuenta con una sólida base metodológica para la identificación, estimación, inferencia y validación \href{https://arxiv.org/abs/2108.09400}{(Cattaneo \& Titiunik, 2022)}. Una de las aplicaciones más comunes de RD es el análisis de temas de economía política en el contexto de elecciones cerradas (\textit{close elections}) \href{https://www.annualreviews.org/content/journals/10.1146/annurev-polisci-032015-010115}{(de la Cuesta \& Imai, 2016)}. En estos estudios, es común evaluar el impacto de características observables de los políticos electos sobre diversas variables de resultado. Por ejemplo, algunos trabajos comparan el desempeño de hombres y mujeres que han obtenido cargos públicos por márgenes estrechos, lo que, en principio, permite aislar el efecto del género mientras se mantienen constantes otras características observables y no observables. \\

En este punto, analizaremos el estudio de \href{https://link.springer.com/article/10.1007/s11109-017-9407-7}{Bucchianeri (2018)}, que examina el impacto de que una mujer, en lugar de un hombre, gane una \textit{``primary election''} sobre sus resultados en las \textit{``general elections''} en Estados Unidos. Las \textit{``primary elections''} son equivalentes a las \textit{consultas internas} en Colombia, en las cuales un partido o coalición selecciona a sus candidatos para las elecciones generales, donde finalmente se eligen los ocupantes de los cargos públicos. \\

La pregunta central de este ejercicio es: \textbf{¿cuál es el efecto de que una mujer gane una consulta interna por un margen estrecho sobre su probabilidad de obtener la victoria en las elecciones generales?} Para ello, considere: \\

\begin{itemize}
    \item $Y_i$ una variable dicótoma que toma el valor de uno si la candidata $i$ ganó la elección general y cero de lo contrario.

    \item $Z_i$ la variable del margen de victoria/derrota de la candidata $i$ en la consulta interna.
    
    \item $D_i$ una variable dicótoma que toma el valor de uno si la candidata $i$ ganó la consulta interna y cero de lo contrario.

    \item $Female_i$ una variable que toma el valor de uno si la candidata $i$ es mujer y cero de lo contrario.

    \item $\textbf{x}_i$ un vector de características observables de la candidata $i$.
\end{itemize}

\begin{enumerate}
    \item De una intuición de por qué el diseño de regresión discontinua es adecuado para responder esta pregunta. Para ello, discuta \textbf{explícitamente} el contexto del estudio y los supuestos de la metodología (\textit{máximo 150 palabras}).
\begin{solucion}

\end{solucion}
    \item Utilizando el lenguaje de resultados potenciales, formalice su respuesta al numeral anterior. En particular,

        \begin{enumerate}        
            \item Formule matemáticamente el problema de identificación.  
            \begin{solucion}
\end{solucion}
            \item Explique por qué, si se cumplen los supuestos, es posible recuperar un efecto causal en este caso. 
            \begin{solucion}
\end{solucion}
        \end{enumerate}

\end{enumerate}

Usted cuenta con una base de datos llamada \textit{female\_politics.dta}, que contiene información sobre elecciones primarias y generales en Estados Unidos desde 1972 hasta 2010. Cada observación corresponde a una combinación de candidata, distrito electoral y año. La base de datos permite identificar si la candidata ganó la consulta interna (\textit{primary\_winner} = 1), el margen de victoria o derrota en la lección primaria (\textit{PrimMargin} $> 0$ indica victoria) y si ganó las elecciones generales (\textit{GenWin} = 1). Además, se incluyen algunas características observables.

\begin{enumerate}[resume]
    \item Genere un gráfico\footnote{Recuerde lo visto en la clase complementaria y siga las instrucciones correspondientes.} de la variable de focalización frente a la variable de interés.

    \begin{enumerate}
    
        \item La discontinuidad (o su ausencia) debe ser claramente observable.  
\begin{solucion}
\end{solucion}
        \item Incluya un polinomio global de grado dos en el gráfico. ¿Por qué podría ser preferible un polinomio de grado dos en lugar de uno de mayor grado?
\begin{solucion}
\end{solucion}
        \item Interprete los resultados obtenidos. Sea explícito en describir qué sucede en el punto de corte. Intuitivamente, ¿cree que estos resultados tienen sentido? (\textit{máximo 300 palabras}).  
        \begin{solucion}
\end{solucion}
    \end{enumerate}

    \item Estime el efecto de que una mujer gane la consulta interna sobre su probabilidad de victoria en las elecciones generales. Para ello, utilice dos enfoques: i) regresión lineal local y ii) polinomio global de grado 2. Para cada enfoque:

    \begin{enumerate}

        \item Explique de manera intuitiva las ventajas y desventajas del método seleccionado (\textit{máximo 150 palabras por propuesta}).
\begin{solucion}
\end{solucion}
        \item Escriba explícitamente la ecuación a estimar.
\begin{solucion}
\end{solucion}
        \item Discuta si es adecuado incluir controles en la regresión. Justifique su decisión (\textit{máximo 150 palabras}).
\begin{solucion}
\end{solucion}
        \item Como prueba de robustez, en la regresión lineal local utilice tres anchos de banda: $BW = 0.15, 0.10$ y $0.05$.
\begin{solucion}
\end{solucion}
        \item Interprete los resultados obtenidos. ¿Qué conclusiones se pueden extraer de este análisis? (\textit{máximo 200 palabras}).
\begin{solucion}
\end{solucion}
    \end{enumerate}

    \item En su investigación, \href{https://link.springer.com/article/10.1007/s11109-017-9407-7}{Bucchianeri (2018)} es muy cuidadoso con la interpretación del efecto estimado: ``[...] es importante aclarar que el diseño de regresión discontinua (RDD) en este caso está identificando el efecto causal de nominar a una candidata [mujer] y no el efecto causal del género.'' 

    \begin{enumerate}
        \item ¿Cuál es la diferencia entre los dos efectos que el autor menciona?
\begin{solucion}
\end{solucion}
        \item ¿Por qué cree usted que el autor descarta la segunda interpretación?
\begin{solucion}
\end{solucion}
        \item Si usted fuera la investigadora, ¿qué efecto consideraría más deseable recuperar? Sustente teniendo en cuenta la \textbf{validez} del diseño experimental y los obstáculos para la identificación.
\begin{solucion}
\end{solucion}        
    \end{enumerate}
  
    
\end{enumerate}

\bigskip
    
\href{https://onlinelibrary-wiley-com.ezproxy.uniandes.edu.co/doi/full/10.1111/ajps.12741}{Marshall (2024)} argumenta que, aunque estimar efectos causales de una \textbf{característica} de un candidato ganador es deseable, no es evidente que un RD pueda aislar dicho efecto usando el método de \textit{close election}. En particular, argumenta que un RD identifica efectos compuestos de muchas características en lugar de un efecto LATE asociado exclusivamente a la característica de interés. \\

En este sentido, mientras un RD tradicional define el tratamiento como caer arriba o abajo del punto de corte, el RD que estamos estudiando en este ejercicio define el tratamiento como tener vs no tener una característica de interés específica (e.g. ser mujer) \textbf{dado} que dicho político ganó una elección cerrada.\footnote{Esto implica que en este tipo de RDs se están comparando políticos que estuvieron en una elección cerrada, pero que difieren en una característica observada dicótoma (e.g. ser mujer).} Frente a esto, el autor señala dos obstáculos importantes: i) las elecciones cerradas no asignan aleatoriamente (ni \textit{as-good-as-random}) la característica de interés y ii) al concentrarse en elecciones cerradas, la característica de interés puede directamente determinar la variable de focalización (el margen de votos). \\

De ahora en adelante, suponga que se desea responder la pregunta: \textbf{¿cuál es el efecto de ser mujer sobre la probabilidad de obtener la victoria en las elecciones generales?}. El objetivo de los siguientes incisos es identificar las principales dificultades para responder este tipo de preguntas en el contexto de los RD.

\bigskip

\begin{enumerate}[resume]
    
\item Explique intuitivamente ambos obstáculos mencionados por \href{https://onlinelibrary-wiley-com.ezproxy.uniandes.edu.co/doi/full/10.1111/ajps.12741}{Marshall (2024)} usando \textbf{explícitamente} el contexto del caso-estudio de \href{https://link.springer.com/article/10.1007/s11109-017-9407-7}{Bucchianeri (2018)}. Mencione por qué atentan contra la identificación del efecto causal. 

%\item Si las mujeres que ganan elecciones cerradas son sistemáticamente distintas a hombres que ganan elecciones cerradas\footnote{Lo son. Por ejemplo, vea \href{https://onlinelibrary.wiley.com/doi/abs/10.1111/j.1540-5907.2011.00512.x}{Anzia \& Berry (2011)} y \href{https://www.cambridge.org/core/journals/politics-and-gender/article/abs/what-it-takes-to-win-questioning-gender-neutral-outcomes-in-us-house-elections/83267D037A804CE682D5A640DA7B27E0}{Pearson \& McGhee (2011)}}, ¿cree usted que esto puede comprometer el efecto que se desea rescatar?
        \begin{solucion}
\end{solucion}
\end{enumerate}

\bigskip

Las preocupaciones de \href{https://onlinelibrary-wiley-com.ezproxy.uniandes.edu.co/doi/full/10.1111/ajps.12741}{Marshall (2024)} pueden implicar un sesgo por un concepto que él llama \textbf{compensadores diferenciales} (\textit{compensating differentials}). Estos se definen como características observadas y/o no-observadas que: 1) son distintas a la característica de interés, y 2) permiten que aunque dos políticos no tengan la misma característica de interés, igualmente se enfrenten en una elección cerrada. \\

Por ejemplo, considere los años de educación de un político. Suponga que los votantes, en igualdad de condiciones, tienen una preferencia general por votar por un hombre en lugar de una mujer, pero a su vez prefieren candidatos con mayor educación. En este caso, la variable de años de educación es un \textbf{compensador diferencial} porque: 1) las variables de años de educación vs ser mujer son teóricamente distintas, y 2) en elecciones cerradas, los hombres van a tener en promedio menos años de educación que las mujeres\footnote{Es decir, los años de educación adicionales que tiene la candidata mujer \textbf{compensa} la desventaja de ser mujer lo suficiente para generar una elección cerrada.}. Note que esto \textbf{no} implica que la variable \textit{años de educación} esté desbalanceada a \textbf{nivel de distrito electoral}. Más bien, indica que, a \textbf{nivel de candidato}, las mujeres que ganan elecciones cerradas difieren de los hombres que también las ganan, más allá de la diferencia de sexo.  \\

En los siguientes incisos, usted va a demostrar que esto implica un sesgo, incluso cuando no hay correlación entre ser mujer y el compensador diferencial. Suponga que ser mujer ($Female_i = 1$) es perjudicial para la probabilidad de ganar una elección general y sea $W_i$ un compensador diferencial que tiene un impacto negativo en la probabilidad de ganar las elecciones generales. \\

Suponga que en una consulta interna fija con dos candidatxs $i$ y $j$, se tiene que:
\vspace{0.2cm}
\begin{equation}
    Votos_i = \beta_F\frac{Female_i-Female_j}{2}+\beta_W\frac{W_i-W_j}{2}+\frac{\epsilon_i-\epsilon_j}{2}
\end{equation}
\vspace{0.2cm}
donde $W_i-W_j \sim \mathcal{N}(0, \sigma_W^2)$ es la resta del compensador diferencial y $\epsilon_i-\epsilon_j \sim \mathcal{N}( 0, \sigma_\epsilon^2)$ es un choque aleatorio. Sea $Y_i^1 = \tau Female_i + \gamma W_i + u_i$ el resultado potencial de que la persona $i$ gane la elección general, con $u_i$ un término de error independiente de todas las variables.

\bigskip

\begin{enumerate}[resume]
    \item Suponga que $Female_i \; \indep \; W_i, W_j$ y que $\tau,\gamma \leq 0$. Demuestre que el estimador de RD está asintóticamente sesgado. Para ello:

    \begin{enumerate}

        \item Explique breve e intuitivamente por qué $Female_i \; \indep \; W_i, W_j$ es un ``best case scenario''.
        \begin{solucion}
\end{solucion}
            
        \item Demuestre que si en una consulta interna que enfrenta a una política $i$ y a un político $j$ ambos reciben exactamente el mismo número de votos, entonces

        \begin{equation*}
            \beta_F + \beta_W(W_i - W_j) + (\epsilon_i - \epsilon_j)  = 0.
        \end{equation*}

        Interprete brevemente los mecanismos que podrían justificar este empate.
\begin{solucion}
\end{solucion}
        \item Demuestre que $\plim_{n \to \infty} \hat{\tau}_{RD} < \tau$. 
        \begin{solucion}
\end{solucion}
        \item ¿Por qué  el estimador de RD subestima el efecto causal de ser mujer? Justifique intuitivamente y de un ejemplo de un \textbf{compensador diferencial} que cumple esto.
        \begin{solucion}
\end{solucion}
        
    \end{enumerate}
    
\end{enumerate}

\bigskip 

Finalmente, \href{https://onlinelibrary-wiley-com.ezproxy.uniandes.edu.co/doi/full/10.1111/ajps.12741}{Marshall (2024)} señala que una estrategia prometedora para abordar este problema es obtener un estimador consistente del LATE del compensador diferencial, de manera que sea posible distinguir qué parte de $\hat{\tau}_{RD}$ corresponde al efecto del compensador diferencial y qué parte representa el efecto de interés.

\bigskip

\begin{enumerate}[resume]
\item Suponga que el único compensador diferencial son los años de educación y que dicha variable corresponde a una característica observable. ¿Cómo implementaría la propuesta de Marshall (2024)? \textbf{No} es necesario ejecutar en Stata la corrección, sólo describir en detalle el procedimiento.
    
    % \begin{enumerate}  
    %     % \item A partir de la figura realizada en la pregunta 3, realice las correcciones necesarias de manera que se pueda evidenciar el sesgo asintótico en su gráfica.  

    %     \item Proponga (pero \textbf{no} implemente) una metodología que permita identificar el efecto causal de ser mujer sobre la probabilidad de ganar las elecciones generales.  
    % \end{enumerate}  
\begin{solucion}
\end{solucion}
\end{enumerate}

\newpage
\section*{Segundo Ejercicio}
El sesgo de selección es el principal obstáculo en la estimación de efectos causales a partir de datos observacionales. Si hay características no observables ($\epsilon_i$) capaces de afectar simultáneamente la variable de interés y la variable de resultado, podemos caer en el error de atribuir efectos causales a simples correlaciones espurias. La técnica de variables instrumentales es uno de los métodos más conocidos para abordar este tipo de problema de endogeneidad. De forma intuitiva, una buena variable instrumental permite aislar la variación exógena de la variable explicativa de interés y utilizar únicamente esta \textit{``variación limpia''} para estimar el efecto de interés. En este punto, analizaremos el uso de variables intrumentales (IV) en contextos no binarios.

\bigbreak
\href{https://www.nber.org/system/files/working_papers/w5778/w5778.pdf}{Angrist \& Evans (1996)} analizan el efecto de tener un hijo adicional sobre la oferta laboral de sus padres. En particular, los autores parten de la premisa de autores como \href{https://www.nber.org/system/files/working_papers/w5188/w5188.pdf}{Goldin (1995)}, que sugiere que el número de hijos es un determinante del desarrollo profesional de los padres, especialmente las mujeres. Esto porque, entre otros, tener más hijos implica una mayor carga de trabajo en el hogar. 

\bigbreak
Los autores cuentan con una base de datos con información a nivel de hogar del número de horas trabajadas por semana ($Y_i$), el número de hijos ($W_i$) en el hogar $i$ y el sexo de cada uno de ellos. Para empezar, los autores proponen estimar la siguiente regresión:

\begin{equation}\label{eq1p2}
    Y_i = \beta_0 + \beta_1 W_i + \epsilon_i
\end{equation}

\begin{enumerate}
    \item[1.] ¿Considera que estimar la ecuación \ref{eq1p2} por MCO permite recuperar el efecto causal de interés? ¿Qué tipo de variables pueden generar un problema de endogeneidad? (\textit{Máximo 150 palabras})
\begin{solucion}

No, la ecuación \ref{eq1p2} no permite recuperar el efecto causal de interés porque se cree que hay un problema de endogeneidad, esto es $\text{cov}(W_i, \epsilon_i) \neq 0$. Formalmente, las variables que  pueden generar endogeneidad son variables tales que 1) se correlacionan con factores no observables que afecten el número de horas trabajadas por el hogar $i$ $(Y_i)$ y 2) que a su vez se correlaciones con el número de hijos por hogar $(W_i)$. Fácilmente, se pueden encontrar varias variables, por ejemplo años de educación de la madre.
    \begin{itemize}
        \item $cov(W_i, edu_i)<0:$ para mujeres más educadas, es más costoso pausar su carrera profesional, por lo que tienen menos hijos.
        \item $cov(Y_i, edu_i)>0:$ para mujeres más educadas, habrá más opotunidades profesionales, asumirán más labores, como juntas directivas, consultorías, etc.
\end{itemize}
Otra variable es un indicador de si el hogar es monoparental. La crianza es más complicada y como la persona a cargo tiene más responsabilidades, se esperaría que tenga menos horas laborales así como un menor número de hijos.

Gráficamente,
\begin{center}
\begin{tikzpicture}[
    every node/.style={draw=none, text width=1.5cm, align=center, font=\small}, 
    every path/.style={thick, black},
    >={Latex[length=3mm, width=2mm]} % Ajusta el tamaño de la punta
]

% Nodos
\node (W) { $W_i$ };
\node (Y) [right=of W] { $Y_i$ };
\node (edu) [below left=of W, align=center] { $edu_i$ \\ $monoparental_i$ };
\node (u) [below right=of W] { $\epsilon_i$ };

% Flechas sólidas con punta grande
\draw[-{Latex[length=3mm, width=2mm]}] (W) -- (Y);
\draw[-{Latex[length=3mm, width=2mm]}] (u) -- (Y);

% Flechas punteadas con punta grande
\draw[dashed, -{Latex[length=3mm, width=2mm]}] (edu) -- (W);
\draw[dashed, -{Latex[length=3mm, width=2mm]}] (edu) -- (u);

\end{tikzpicture}
\end{center}
\end{solucion}
\end{enumerate}

\bigbreak
Los autores proponen utilizar como instrumento de \(W_i\) una variable dicótoma que toma el valor de uno si todos los hijos del hogar son del mismo sexo, y cero en caso contrario. La lógica es que los padres cuyos hijos son todos del mismo sexo tienen una mayor probabilidad de intentar tener un hijo adicional dadas las preferencias de los hogares por tener al menos un hijo de sexo diferente. Al mismo tiempo, el sexo de los hijos está determinado principalmente por factores biológicos, fuera del control de los padres. 

\begin{enumerate}
    \item[2.] Sea \(Z_i\) una variable dicótoma que toma el valor de uno si todos los hijos del hogar \(i\) en línea base tienen el mismo sexo, y cero en caso contrario. Note que este instrumento solo está definido en hogares con dos o más hijos de manera que es posible comparar el sexo entre ellos. Suponga en todo este punto que solo analizaremos hogares con al menos dos hijos en línea base; es decir, con $W_i \geq 2$.

    \bigbreak
    Bajo el enfoque LATE, liste los supuestos que se necesitan para recuperar un efecto causal a través de la estrategia de Variables Instrumentales. ¿Cree usted que se cumplen en la estrategia de identificación planteada por \href{https://www.nber.org/system/files/working_papers/w5778/w5778.pdf}{Angrist \& Evans (1996)}?. (\textit{Máximo 250 palabras})  

\begin{solucion}
Primero, definimos los resultados potenciales par la variable de resultado: 
\begin{equation*}
    Y_i^{W,Z} = \begin{cases}  Y_i^{2,0} & \text{horas que trabaja el hogar} \: i  \text{ si tiene 2 hijos en línea base  de diferente sexo}  \\Y_i^{2,1} & \text{horas que trabaja el hogar} \:i \text{ si tiene 2 hijos en línea base  del mismo sexo} \\
    \vdots\\
     Y_i^{J,0} & \text{horas que trabaja el hogar} \:i \:\text{si tiene J hijos en línea base  de diferente sexo} \\ Y_i^{J,1} & \text{horas que trabaja el hogar}\: i \: \text{si tiene J hijos en línea base  de diferente sexo}
     \end{cases}
\end{equation*}

Supuestos:
\begin{enumerate}
    \item \textbf{Independencia:}  Los resultados potenciales de $Y_i$ y de $W_i$ son independientes de $Z_i$. \\
    \textbf{Cumplimiento:} Sí. El sexo de los hijos está fuera del control de los padres, entonces el instrumento es as good as random.
    
    
    \item \textbf{Restricci\'on de Exclusi\'on:}  $Z_i$ solo afecta a $Y_i$ a trav\'es de $W_i$
     \\
    \textbf{Cumplimiento:} No necesariamente. La razón es que a pesar de que un hijo es una 'boca' que alimentar independiente del sexo; cuando el hogar tiene solo hijas, es probable que la madre sienta una motivación adicional a trabajar fuertemente para darle ejemplo a sus hijas sobre el empoderamiento femenino y, mostrarles que es posible realizarse como madre y ser exitosa profesionalmente. Entonces en este caso $Y_i^{W,0} \neq Y_i^{W,1},\: W=2,..., J$
    
    \item \textbf{Relevancia:}  $Z_i$ tiene influencia sobre $W_i$. 
     \\
    \textbf{Cumplimiento:} Sí. Es bien sabido popularmente, que las familias tienen una preferencia por tener hijos de diferentes sexos. Entonces sí hay una correlación (positiva) entre el instrumento y el número de hijos.
    

    \item \textbf{Monotonicidad:} Si el instrumento afecta la decisi\'on sobre el número de hijos, entonces la afecta en la misma direcci\'on para todos los individuos (i.e., No hay defiers). \\
    \textbf{Cumplimiento:} Sí, es de esperarse que $ W_i^1 - W_i^0 \geq 0 \quad \forall i$. Se espera que el número de hijos que tendría el hogar si sus hijos tuviesen el mismo sexo en linea base sea mayor al número de hijos que tendría si sus hijos tuviesen diferente sexo en linea base.
\end{enumerate}
\end{solucion}
\end{enumerate}

% En la interpretación moderna de Variables Instrumentales, el efecto local promedio del tratamiento (LATE) se puede estimar a través de Mínimos Cuadrados Ordinarios en Dos Etapas (TSLS) o a través del Estimador de Wald, que en su versión poblacional toma la forma: 

%     \begin{equation} \label{wald}
%         \frac{E[Y_i| Z_i = 1] - E[Y_i| Z_i = 0]}{E[Y_i| W_i = 1] - E[Y_i| W_i = 0]} 
%     \end{equation}

%     \hspace{2cm} con $Y_i^1 - Y_i^0 = \tau_i$

% \bigbreak
En los ejemplos de la clase mostramos que el Estimador de Wald en su versión poblacional, cuando el instrumento y la variable de interés es binaria, toma la forma: 

    \begin{equation} \label{wald}
        \frac{E[Y_i| Z_i = 1] - E[Y_i| Z_i = 0]}{E[Y_i| W_i = 1] - E[Y_i| W_i = 0]} = E[\tau_i | W_i^1 > W_i^0]
    \end{equation}


\bigbreak
Lo que implica que el efecto estimado aplica a una subpoblación particular, que llamamos \textit{compliers}. Sin embargo, en casos como el propuesto por \href{https://www.nber.org/system/files/working_papers/w5778/w5778.pdf}{Angrist \& Evans (1996)}) en que la variable es discreta o continua, la interpretación cambia. En los siguientes puntos analizaremos cómo cambia la interpretación del efecto estimado cuando la variable endógena de interés es discreta.

\begin{enumerate}
    \item[3.] Definamos $W_i$ formalmente como una variable multivariada discreta $W \in \{2,...,J\}$ con $J-1$ niveles. Considere que $W_i$ es el número de hijos del hogar $i$.\footnote{\footnotesize{Recuerde que solo estamos considerando hogares con al menos dos hijos.}} Además, definamos $W_i^z$ como el resultado potencial cuando el instrumento toma el valor $z$ (es decir, si $Z_i = z$).\footnote{\footnotesize{Por ejemplo, $W_i^1$ sería el número de hijos de hogar $i$ si este hogar tuviera solo hijos de un solo sexo (i.e. $Z_1 == 1$)}} Con base en lo anterior:

    
        \begin{enumerate}

            \item Presente una función con los posibles estados potenciales de $W_i$ en función del instrumento $Z_i$. Además presente una breve descripción de cada uno de ellos. (\textit{Máximo 100 palabras}).

            \begin{solucion}
            La función con los estados potenciales para $W_i$ en función de $Z_i$ está dada por:
            \begin{equation*}
    W_i = \begin{cases} W_i^1, & \text{si } Z_i=1  \\ W_i^0, & \text{si } Z_i=0  \end{cases}
\end{equation*}
donde 
\begin{itemize}
    \item  $W_i^1:$ número de hijos que tendría el hogar $i$ si tuviese solo hijos del mismo sexo en línea base 
    \item  $W_i^0:$ número de hijos que tendría el hogar $i$ si tuviese solo hijos de diferente sexo en línea base 
\end{itemize}
\end{solucion}

           Para derivar la expresión del Estimador de Wald en contextos donde \( W_i \) no es binaria, es necesario identificar los equivalentes poblacionales tanto del numerador como del denominador del estimador. \href{https://www.jstor.org/stable/2291054}{Angrist \& Imbens (1995)}} demostró que, cuando la variable endógena de interés \( W_i \) es no binaria, el numerador del Estimador de Wald se expresa como:


            \begin{equation}
                \label{numerador}
                 E[Y_i | Z_i = 1 ] - E[Y_i | Z_i = 0 ] = \sum_{j=2}^J \Big( E[\tau_i | W_i^1 \geq j > W_i^0]  \times  Pr[W_i^1 \geq j > W_i^0] \Big) 
            \end{equation} 

            \item Asuma el resultado anterior.\footnote{\footnotesize{Si usted desea conocer la prueba de la ecuación \ref{numerador} visite el artículo de \href{https://www.jstor.org/stable/2291054}{Angrist \& Imbens (1995)}} o \href{https://ideas.repec.org/p/fri/fribow/fribow00492.html}{Eckhof \& Huber (2018)}} Explique en sus palabras qué quiere decir el condicional $W_i^1 \geq j > W_i^0$. En particular, ¿cuáles son las condiciones de los hogares que cumplen con este criterio? (\textit{Máximo 150 palabras}).
            \begin{solucion}
            $W_i^1 \geq j $: dice que el numero de hijos que tendría el hogar dado que en linea base eran del mismo sexo (Z==1), es mayor a $j$.\\
            $j > W_i^0 $: dice que el numero de hijos que tendría el hogar dado que en linea base eran de diferente sexo (Z==0), es menor a $j$.\\

            Entonces $W_i^1 \geq j > W_i^0$ representa a los hogares del mismo sexo, que empezaron a tener mas hijos a fin de tener hijos de diferente sexo y que en el hijo j esimo lograron teenr el hijo de diferente sexo....\\
            
            la cantidad de hijos adicionales que el hogar $i$ empezó a tener a fin de tener un hijo de otro sexo. Note que si el hogar ya tenía hijos de diferente sexo en línea base, entonces no importa, no lo está contando. Pero si el hogar tenia hijos del mismo sexo en línea base, entonces hasta el hijo $j-1$ tenian hijos del mismo sexo y justo en $j$ ya tienen un hijo de un sexo diferente. Ahora, puede que en línea base hayan tenido menos de $j-1$ hijos y que hayan intentado y les haya salido hijos del mismo sexo.\\

             ¿cuáles son las condiciones de los hogares que cumplen con este criterio? hogares que tenian hijos de diferente sexo en linea base\textcolor{red}{MIRAR ESTO REVISAR!!}
            
           
\end{solucion}
            \item Ahora, utilice la definición de valor esperado condicional para demostrar que en el contexto no binario, el denominador del Estimador de Wald es igual a:
            
            \begin{align}
                E[W_i | Z_i = 1 ] - E[W_i | Z_i = 0 ] = \sum_{j=2}^J \Big( Pr[W_i^1 \geq j > W_i^0] \Big) 
            \end{align}

            Para esta demostración asuma además que los supuestos que listó en el numeral 2 se cumplen. 

          \begin{solucion}
\begin{proof}
El valor esperado condicional de $W_i$ dado $Z_i = z$ se define como:

\begin{align*}
    E[W_i \mid Z_i = 1] = \sum_{j=2}^{J} j \cdot Pr(W_i = j \mid Z_i = 1)
\end{align*}
A su vez,
\begin{align*}
    E[W_i] = \sum_{j=2}^{J} j \cdot Pr(W_i = j )
\end{align*}
Entonces:
          \begin{align*}
    E[W_i \mid Z_i = 1] - E[W_i \mid Z_i = 0] &= E[W_i^1 \mid Z_i = 1] - E[W_i^1 \mid Z_i = 0]\\&= E[W_i^1 ] - E[W_i^0] \quad  \text{(Por supuesto de independencia )}\\
    &=\sum_{j=2}^{J} j \cdot Pr(W_i^1 = j ) - \sum_{j=2}^{J} j \cdot Pr(W_i^0 = j ) \\
    &= \sum_{j=2}^{J} j \cdot \left( Pr(W_i^1 = j ) - Pr(W_i^0 = j ) \right)  \\  
    \end{align*}
Ahora, por el supuesto de monotonicidad, note que podemos reescribir la probabilidad de estar exactamente en $j$ como la diferencia entre la probabilidad de ser mayor o igual a $j$ y mayor o igual a $j+1$. Es decir, $Pr(W_i^z = j )= Pr(W_i^z \geq j)-Pr(W_i^z \geq j+1)$, entonces:
    \begin{align*}
    &=\sum_{j=2}^{J} j \cdot(  Pr(W_i^1 \geq j) - Pr(W_i^1 \geq j+1) -   Pr(W_i^0 \geq j)  +   Pr(W_i^0 \geq j+1)) 
    \end{align*}
    Reordenando:
    \begin{align*}
&= \sum_{j=2}^{J} j \cdot \left( Pr(W_i^1 \geq j) - Pr(W_i^0 \geq j) - (Pr(W_i^1 \geq j+1) - Pr(W_i^0 \geq j+1)) \right) \\
&= \sum_{j=2}^{J} \left( j \cdot Pr(W_i^1 \geq j) - j \cdot Pr(W_i^0 \geq j) - j \cdot Pr(W_i^1 \geq j+1) + j \cdot Pr(W_i^0 \geq j+1) \right) \\
\end{align*}
Expandiendo la suma, tenemos que hay términos similares que se cancelan (es suma telescópica):
\begin{align*} 
&\textcolor{orange!50}{2 \cdot (Pr(W_i^1 \geq 2) - Pr(W_i^0 \geq 2))} - \textcolor{blue!50}{2 \cdot (Pr(W_i^1 \geq 3) - Pr(W_i^0 \geq 3))} \\
&+ \textcolor{blue!50}{3 \cdot (Pr(W_i^1 \geq 3) - Pr(W_i^0 \geq 3))} - \textcolor{purple!50}{3 \cdot (Pr(W_i^1 \geq 4) - Pr(W_i^0 \geq 4))} \\
&+ \textcolor{purple!50}{4 \cdot (Pr(W_i^1 \geq 4) - Pr(W_i^0 \geq 4))} - \textcolor{tea!50}{4 \cdot (Pr(W_i^1 \geq 5) - Pr(W_i^0 \geq 5))} \\
&+ \cdots + \textcolor{pink!90}{(J-2) \cdot (Pr(W_i^1 \geq J-2) - Pr(W_i^0 \geq J-2))} \\
&- \textcolor{green!50}{(J-2) \cdot (Pr(W_i^1 \geq J-1) - Pr(W_i^0 \geq J-1))} \\
&+ \textcolor{green!50}{(J-1) \cdot (Pr(W_i^1 \geq J-1) - Pr(W_i^0 \geq J-1))} \\
&- \textcolor{red!50}{(J-1) \cdot (Pr(W_i^1 \geq J) - Pr(W_i^0 \geq J))}\\
&+ \textcolor{red!50}{J \cdot (Pr(W_i^1 \geq J) - Pr(W_i^0 \geq J))} \\
&- \textcolor{brown!50}{J \cdot (Pr(W_i^1 \geq J+1) - Pr(W_i^0 \geq J+1))}
\end{align*}
Note dos cosas para el primer y el último término:
\begin{itemize}
    \item $\textcolor{orange!50}{2 \cdot (Pr(W_i^1 \geq 2) - Pr(W_i^0 \geq 2))} =0$ porque ambas probabilidades son 1 (por construcción del ejercicio cada hogar tiene mínimo 2 hijos), luego la resta es 0.
     \item $\textcolor{brown!50}{J \cdot (Pr(W_i^1 \geq J+1) - Pr(W_i^0 \geq J+1))}=0$  porque ambas probabilidades son 0 (por construcción del ejercicio cada hogar tiene máximo $J$ hijos), luego la resta es 0.
\end{itemize}
Ahora, para los términos restantes, cancelando los términos (el último de cada fila con el primero de la fila siguiente de manera respectiva):
\begin{align*}
&= (Pr(W_i^1 \geq 2) - Pr(W_i^0 \geq 2) )+ ((Pr(W_i^1 \geq 3) - Pr(W_i^0 \geq 3)) \\
&+ \cdots + ((Pr(W_i^1 \geq J) - Pr(W_i^0 \geq J))\\
&= Pr(W_i^1 \geq 2 > W_i^0) + Pr(W_i^1 \geq 3 > W_i^0) + \cdots + Pr(W_i^1 \geq J > W_i^0)\\
&= \sum_{j=2}^{J} Pr(W_i^1 \geq j > W_i^0)
\end{align*}
Por tanto, \begin{align*}
                E[W_i | Z_i = 1 ] - E[W_i | Z_i = 0 ] = \sum_{j=2}^J \Big( Pr[W_i^1 \geq j > W_i^0] \Big) 
            \end{align*}
\end{proof}

\end{solucion}
                \item Utilizando los dos resultados anteriores, muestre que: 

        \begin{equation}\label{eq2p2}
            \frac{E[Y | Z_i = 1] - E[Y | Z_i = 0]}{E[W | Z_i = 1] - E[W | Z_i = 0]} = \sum_{j = 2}^J w_j E\Bigg[\tau_i |  W_i^1 \geq j \geq W_i^0 \Bigg]
        \end{equation}

        con\footnote{\footnotesize{También se puede demostrar que $0 \leq w_j \leq 1$ y $\sum_{j=2}^J w_j = 1$. Esto, solo para su conocimiento, no hace falta que lo pruebe.}} $w_j = \frac{P(W_i^1 \geq j \geq W_i^0 )}{\sum_{j=1}^J P(W_i^1 \geq j \geq W_i^0 )}$ 
        \begin{solucion}
        \begin{proof}
        Del punto b) sabemos que 
         \begin{equation*}
                 E[Y_i | Z_i = 1 ] - E[Y_i | Z_i = 0 ] = \sum_{j=2}^J \Big( E[\tau_i | W_i^1 \geq j > W_i^0]  \times  Pr[W_i^1 \geq j > W_i^0] \Big) 
            \end{equation*}
        Del punto c) sabemos que 
        \begin{equation*}
                E[W_i | Z_i = 1 ] - E[W_i | Z_i = 0 ] = \sum_{j=2}^J \Big( Pr[W_i^1 \geq j > W_i^0] \Big)
            \end{equation*}
       Entonces combinando esto
        \begin{align*}
            \frac{E[Y | Z_i = 1] - E[Y | Z_i = 0]}{E[W | Z_i = 1] - E[W | Z_i = 0]}  &= \frac{\sum_{j=2}^J \Big( E[\tau_i | W_i^1 \geq j > W_i^0]  \times  Pr[W_i^1 \geq j > W_i^0] \Big) }{\sum_{j=2}^J \Big( Pr[W_i^1 \geq j > W_i^0] \Big)}
        \end{align*}
        Defina $w_j = \frac{P(W_i^1 \geq j \geq W_i^0 )}{\sum_{j=2}^J P(W_i^1 \geq j \geq W_i^0 )}$. Entonces:
        \begin{align*}
                & = \sum_{j = 2}^J  E\Bigg[\tau_i |  W_i^1 \geq j \geq W_i^0 \Bigg] w_j
            \end{align*}
        \end{proof}
\end{solucion}

        
        \item Interprete el efecto que recupera la parte derecha del Estimador de Wald hallado arriba. ¿Sobre qué subpoblación se calcula el efecto? ¿Qué observaciones se ponderan más? ¿Cuáles menos? (\textit{Máximo 200 palabras})
        \\ \\ \textbf{Pista:} Note que $j \geq W_i^0$ le está hablando de hogares para los cuales el número de hijos que habrían tenido si el sexo de sus hijos en baseline fuera diferente (i.e. si $Z_i = 0$) es menor que un número dado $j \in \mathbbm{R}$. $W_i^1 \geq j$ se interpreta de manera similar.
        \end{enumerate}

      \begin{solucion}
      \begin{itemize}
          \item   Interpretación del efecto:
          \item Subpoblación:El condicional esta sobre los hogares en donde  $W_i^1  > W_i^0$. Es decir, unicamente se consideran los hogares en donde el instrumento tiene el efecto esperado, este es el grupo de compliers. Ahora, por el supuesto de monotonicidad, uno esperaria que todos los hogares estudiados pertenenciera a este grupo, es decir que no tenemos defiers (o sea tener hijos del mismo sexo hace que ya no quieran tener mas hijos, pero tener hijos de diferente sexo hace que quieran tener mas hijos; claro eso podría pasar dependiendo de las dinamicas del hogar pero es otra discusión)
          \item   Ponderación de observaciones: Note que sí hay como una especie de doble contabilidad. 

          $P(W_i^1 \geq j \geq W_i^0 )=P(W_i^1 \geq j )- P(W_i^0 \geq j) $.\\
          $P(W_i^1 \geq j )$ =la probabilidad de que el número de hijos del hogar $i$ sea mayor que $j$ dado que en línea base tenía hijos del mismo sexo.\\
          $P(W_i^0 \geq j )$ =la probabilidad de que el número de hijos del hogar $i$ sea mayor que $j$ dado que en línea base tenía hijos de diferente sexo.\\
          
          Luego la diferencia $P(W_i^1 \geq j )- P( W_i^0 \geq j)$ 
          mide cuánto más probable es que un hogar tenga al menos 
$j$ hijos si en la línea base tenía hijos del mismo sexo, en comparación con hogares que tenían hijos de sexos diferentes.\\

$P(W_i^1 \geq j \geq W_i^0 )$ mide la diferencia marginal entre las dos probabilidades condicionales, es decir, indica el cambio en la probabilidad de tener al menos $j$ hijos cuando se pasa de una condición base a otra. Por ejemplo, si esa probabilidad es x, significa que las familias con hijos del mismo sexo tienen una probabilidad x puntos porcentuales mayor de llegar a al menos j hijos en comparación con las familias con hijos de sexos diferentes en linea base.
\textcolor{red}{Revisar}
      
      \end{itemize}
  
     

     
      
      Este efecto está calculado sobre los hogares que en línea base tenían hijos del mismo sexo. El condicional delimita que la población de interés son los hogares que el momento 

      
\end{solucion}
   \end{enumerate}
        
    Un problema de la estimación ofrecida por la ecuación \ref{eq2p2} es que esta puede considerar a un mismo \textit{complier} más de una vez. En particular, note que un hogar puede satisfacer tanto $W_i^1 \geq j > W_i^0$ como $W_i^1 \geq j + 1 > W_i^0$. En este caso, el estimador de Wald pondera al mismo hogar múltiples veces.
    
    \begin{enumerate}
        \item[4.] ¿Qué tipo de problemas puede generar el doble o múltiple conteo de unidades en la estimación? ¿Qué tipo de retos genera en el uso de IV en contextos no binarios como este? (\textit{Máximo 150 palabras}).
        
        
\begin{solucion}
problema de sesgo? sobrestimacion? generar sesgos en la interpretación del estimador de Wald, ya que ciertos hogares recibirían un peso excesivo en la estimación del efecto causal\textcolor{red}{COMPLETAR}
\end{solucion}
        
    \end{enumerate}

    \newpage

\section*{Tercer Ejercicio}


Muchos estudios analizan situaciones en las que las unidades están expuestas de manera diferenciada a un conjunto de choques comunes. Por ejemplo, \href{https://www.aeaweb.org/articles?id=10.1257/aer.103.6.2121}{Autor et al. (2013)} estudian cómo el aumento de las importaciones chinas entre 2000 y 2011 afectó los mercados laborales locales en Estados Unidos. En ese estudio, la exposición regional al denominado “China shock” se mide a partir del porcentaje de empleo local en industrias que se vieron particularmente afectadas por la competencia de China. El hecho que la exposición se define utilizando datos de un período base anterior al choque, implica que se está capturando la composición industrial preexistente de cada región. \\

La idea fundamental detrás de esta metodología es capturada por el instrumento \textit{shift-share}. Instrumentos de este estilo agregan un conjunto común de choques (shifts) con pesos heterogéneos de exposición (shares). Estos instrumentos se emplean ampliamente en campos como economía del trabajo, comercio, macroeconomía, finanzas, etc. Si bien se remontan al menos hasta Freeman (1975, 1980), y fueron popularizados por \href{https://ideas.repec.org/b/upj/ubooks/wbsle.html}{Bartik (1991)}, el número de estudios recientes que los utiliza ha crecido notoriamente en los últimos diez años \href{https://arxiv.org/abs/2405.20604}{(Goldsmith-Pinkam, 2024)}. La idea del siguiente ejercicio será familiarizarnos con este tipo de variables instrumentales, analizarlas críticamente, y aplicarlas a un estudio económico real. \\

Suponga que la variable de resultado, \(y_i\), representa el crecimiento del empleo en la región \(i\) medido en un período posterior al choque (de 2000 a 2011). Siguiendo el lenguaje de evaluación de impacto, consideramos que la variable explicativa \(x_i\) es el tratamiento, que en este caso mide el cambio en el valor de las importaciones provenientes de china durante el mismo periodo. En el estudio de Autor et al. (2013), el instrumento es de tipo shift-share. Específicamente, la exposición de la región se mide combinando dos componentes fundamentales. Por un lado, los \emph{shares} \(s_{ik}\) son las proporciones de empleo que la región \(i\) tenía en cada industria \(k\) en un período base anterior (por ejemplo, en 1995), lo que refleja su estructura industrial preexistente. Por otro lado, los \emph{choques} o \emph{shifts} \(g_k\) se definen como los cambios en las importaciones chinas \textit{a nivel nacional} para cada industria\footnote{En el artículo, esta medida se construye usando el cambio sectorial de las importaciones chinas para otros países de ingreso alto. Aunque este detalle es importante, vamos a omitirlo en el ejercicio.} \(k\) durante el período del ``China shock'' 2000–2011. Al combinar ambos componentes, el instrumento shift-share se define como
\[
z_i = \sum_{k=1}^{K} s_{ik}\,g_k, \label{eq:shift-share} \tag{1}
\]
lo que implica que una región con una alta concentración de industrias que experimentaron un fuerte aumento en las importaciones chinas tendrá un mayor valor de \(z_i\). Este valor refleja, en esencia, la mayor exposición al choque –o mayor competencia–, y se utiliza como instrumento para \(x_i\) en la estimación del efecto causal sobre \(y_i\). \\

Nuestro objetivo será utilizar \(z_i\) como instrumento para el tratamiento \(x_i\) en el modelo estructural

\[
y_i = \alpha +\beta\,x_i + \gamma' w_i + \varepsilon_i,
\]

donde \(w_i\) es un vector de controles exógenos, y \(\varepsilon_i\) es el término de error estocástico.

\begin{enumerate}
    \item[1] Interprete la expresión (\ref{eq:shift-share}) en sus propias palabras. Apóyese de un ejemplo de la vida real diferente al discutido en Autor et al. (2013)
    
    
    \textit{Pista: Asegúrese de entender la estructura del instrumento antes de continuar con el ejercicio.}
\begin{solucion}
\end{solucion}

    \item[2] Para que $z_i$ sea un instrumento válido, debe cumplir la restricción de exclusión $E[z_i \varepsilon_i] = 0$. A continuación utilizaremos la estructura de (\ref{eq:shift-share}) para proponer tres argumentos distintos a partir de los cuales $z_i$ sería un instrumento valido para el modelo estructural. \\
        
        \textit{Pista: Reemplace la estructura del instrumento en la restricción de exclusión}
    \begin{itemize}
        \item[a)] Suponga que $\mathbb{E}[s_{ik} \varepsilon_i]=\mathbb{E}[g_{ik} \varepsilon_i]=0$. ¿Cumpliría $z_i$ la restricción de exclusión? Explique e interprete.
        \begin{solucion}
\end{solucion}
        \item[b)] Suponga que $\mathbb{E}[s_{ik}\varepsilon_i] = 0$, pero que $\mathbb{E}[g_k\varepsilon_i] \neq 0$. ¿Cumpliría $z_i$ la restricción de exclusión? Describa e interprete brevemente bajo qué condiciones se presenta este caso.
        \begin{solucion}
\end{solucion}
        \item[c)] Suponga que $\mathbb{E}[g_k\varepsilon_i] = 0$ pero que $\mathbb{E}[s_{ik}\varepsilon_i] \neq 0$. ¿Cumpliría $z_i$ la restricción de exclusión? Describa e interprete brevemente bajo qué condiciones se presenta este caso.
        \begin{solucion}
\end{solucion}

    \end{itemize}
\end{enumerate}

\href{https://academic.oup.com/restud/article-abstract/89/1/181/6294942?redirectedFrom=fulltext}{Borusyak, Hull y Jaravel (2021)} argumentan que instrumentos como $z_i$ pueden ser plausiblemente exógenos desde el lado de los \textit{shifts}, es decir, siempre y cuando $g_k$ pueda considerarse como exógeno. El argumento de los autores es que incluso si los shares son endógenos, una suma ponderada de los shifts sigue siendo exógena. \\

Denote $y^{\perp}_{i}$ como el residual de proyectar linealmente $y_i$ sobre el vector de controles $w_i$, y similarmente, $x^{\perp}_{i}$ como el residual de proyectar $x_i$ sobre $w_i$. 

\begin{enumerate}
    \item[3] Use el teorema de Frisch-Waugh-Lovell para demostrar que el estimador de mínimos cuadrados en dos etapas (2SLS) puede expresarse como:
    \begin{align}
    \hat{\beta} = \frac{\sum_{i} z_i y^{\perp}_{i}}{\sum_{i} z_i x^{\perp}_{i}}
    \end{align}

    \begin{solucion}
\end{solucion}

\item[4] Usando el resultado anterior, demuestre que el estimador de SSIV (\textit{shift-share instrumental variable}) $\hat{\beta}$ es equivalente al coeficiente de la segunda etapa de una regresión IV a nivel de choque ponderada por $s_{ik}$ que utiliza los choques $g_k$ como instrumento en la estimación de 
    $$ \bar{y}^\perp_k = \beta \bar{x}^\perp_k + \bar{\varepsilon}^\perp_k, $$

    donde la notación $\bar{v}_k = \frac{\sum_i s_{ik} v_i}{\sum_i s_{ik}}$ representa un promedio ponderado por la exposición de la variable $v_i$. Interprete sus resultados.

    \begin{enumerate}[label=\alph*)]
        \item Escriba la expresión para el estimador SSIV $\hat{\beta}_{\text{SSIV}}$ utilizando el instrumento $z_i = \sum_k s_{ik}g_k$ a nivel de región $i$.
        \begin{solucion}
\end{solucion}
        
        \item Escriba la expresión para el estimador IV $\hat{\beta}_{\text{(shock-level)}}$ a nivel de choque utilizando $g_k$ como instrumento.
        \begin{solucion}
\end{solucion}
        
        \item Demuestre que $\sum_k g_k\bar{y}^\perp_k = \sum_i z_i y_i^\perp$ manipulando la definición del promedio ponderado y cambiando el orden de las sumatorias. Asimismo, suponga que los shares suman a $1$, es decir $\sum_i s_{ik} = 1$.  \textit{(Pista: utilice la definición de $\bar{y}^\perp_k$ y de $z_i$)}.

        \begin{solucion}
\end{solucion}
        \item Utilizando su resultado de la parte (c), demuestre que $\hat{\beta}_{\text{SSIV}} = \hat{\beta}_{\text{(shock-level)}}$.
        \begin{solucion}
\end{solucion}
        \item Interprete este resultado de equivalencia. ¿Qué nos dice sobre la validez del enfoque SSIV? ¿Cuáles son las implicaciones para los investigadores que utilizan instrumentos de shift-share y que argumentan exogeneidad por el lado de los \textit{shifts}?
        \begin{solucion}
\end{solucion}
            \end{enumerate}

\end{enumerate}
\href{https://academic.oup.com/restud/article-abstract/89/1/181/6294942?redirectedFrom=fulltext}{Borusyak, Hull y Jaravel (2021)} proponen que el supuesto de identificación clave de esta metodología radica en que los \textit{shifts} $g_k$ deben no estar correlacionados con un promedio de $\varepsilon_i$ de las unidades con pesos $s_{ik}$. Formalmente,

\begin{align}
     \mathbb{E}[g_k | \bar{\varepsilon}_k, s] = \mu \quad \forall k.
     \label{eq:ssiv_assumption}
\end{align}

\begin{enumerate}
    \item[5] Interprete y discuta la validez de (\ref{eq:ssiv_assumption}) en el contexto estudiado por Autor et al. (2013). ¿Es plausible? Argumente su respuesta
    \begin{solucion}
\end{solucion}
    
    \item[6] Demuestre que bajo (\ref{eq:ssiv_assumption}) $z_i$ es un buen instrumento para $x_i$. Es decir, pruebe que $\mathbb{E}[z_i \varepsilon_i]=0$ y que $\mathbb{E}[z_ix_i]\neq 0$.\footnote{Para ello, puede asumir que $\sum_i^K s_{ik} = 1$, es decir, que el instrumento puede ser interpretado como un promedio ponderado por los \textit{shares}. }
    \begin{solucion}
\end{solucion}
\end{enumerate}
    
Una estrategia diferente para lograr la identificación con un instrumento de tipo \textit{shift-share} es argumentar la exogeneidad del instrumento por el lado de los \textit{shares}. \href{https://paulgp.com/papers/bartik_gpss.pdf}{Goldsmith-Pinkham (2020)} argumentan que pensando en el conjunto de $s_{ik}$ como \textit{as-good-as-randomly} asignados a las unidades (siendo extraidos como de una loteria) es posible satisfacer la restricción de exclusión. 

\begin{enumerate}
    \item[7] Plantee la condición de momento que debe satisfacerse para que, desde la perspectiva de los \textit{shares}, un SSIV sea un  instrumento válido
    \begin{solucion}
\end{solucion}

    \item[8] Discuta la plausibilidad del cumplimiento del supuesto planteado en el inciso anterior. Puede ayudarse de un ejemplo concreto para desarrollar su argumento.
    \begin{solucion}
\end{solucion}
    
    \item[9] Realice una breve comparación entre las argumentaciones de exogeneidad de un SSIV (i.e., compara el \textit{shares}-approach vs el \textit{shift}-approach). Presente su comparación en la siguiente tabla.

\begin{table}[H]
\centering
\caption{Resumen de Instrumentos Shift-Share}
\label{tab:takeaways}
\begin{tabular}{p{6cm} p{6cm} p{6cm}}
\hline
 & \multicolumn{2}{c}{\textbf{Estrategia}} \\
\cline{2-3}
 & \multicolumn{1}{c}{\textbf{Shifts}} & \multicolumn{1}{c}{\textbf{Shares}} \\
\hline
\textbf{Estrategia de Identificación} 
& 
&  \\
\textbf{Supuestos de Identificación} 
& 
&  \\
\textbf{Pruebas de Validez} 
&  
&  \\
\hline
\end{tabular}
\end{table}
\begin{solucion}
\end{solucion}
\end{enumerate}

\subsection*{Parte práctica}
\href{https://davidcard.berkeley.edu/papers/immigration-and-inequality.pdf}{Card (2009)} investiga la relación que existe entre la inmigración y la desigualdad salarial. Para lograr este objetivo, el ahora premio Nobel estudió cómo distintos flujos de inmigración que tuvieron lugar entre 1980 y 2000 incidieron en le estructura salarial de distintas ciudades en Estados Unidos. Estos flujos fueron heterogéneos en varias dimensiones, incluyendo el lugar de procedencia de los migrantes como su nivel de calificación. En el estudio, el autor hizo uso de la base de datos ``Card.dta'', la cual amablemente compartió con ustedes jóvenes promesas de la econometría. Así mismo, les envió dos archivos de ayuda: ``Codebook.txt'', el cual contiene el directorio de las variables incluidas en la base; y, ``country\_ids.xslx'', el cual contiene la codificación de los países de procedencia de los migrantes. En esencia, esta base de datos contiene los flujos de inmigrantes a 124 ciudades de Estados Unidos durante el periodo 1980-2000, junto con algunas características de estos lugares, tales como la población de inmigrantes que residía en 1980 (antes del gran éxodo), su tamaño y su desarrollo industrial. \\

La relación que se explora en este artículo está dada por

$$w_{lj}=\beta_0+\tau \, \ln (r_{lj})+\beta_2 \boldsymbol{X}_l+\epsilon_{lj}$$

donde $l$ es un indicador de la ciudad y $j$ indexa un subpoblación caracterizada por su nivel educativo. Asimismo, $w_{lj}$ es la brecha salarial (en logs) residual entre inmigrantes y nativos en el grupo $j$\footnote{ Más precisamente, es la diferencia entre el logaritmo del salario residual promedio de los inmigrantes y el logaritmo salario resiudal promedio de los nativos. Para obtener el componente residual, estimamos modelos de regresión lineal en cada población en aras de aplicar el método de ``partialling-out'' para separar el componente del salario que no depende de características del individuo tales como su edad, su sexo, etc.}, $r_{lj}$ es el ratio entre la cantidad de horas de trabajo de inmigrantes y nativos  observadas en el grupo $j$ en la ciudad $l$. Finalmente, $\boldsymbol{X}_l$ es una serie de controles a nivel de ciudad. En este trabajo, nos interesa determinar la influencia que tiene un incremento en el número de horas trabajadas por extranjeros relativo a aquellas provistas por los locales en la diferencias salariales en dos niveles: cuando los trabajadores cuentan sólo con un grado de educación secundaria, versus cuando tienen un grado equivalente a un título universitario. Así las cosas, los grupos de interés son $j \in \{hs,coll\}$ , esto es, la población que cuenta a lo sumo con un grado de educación secundaria ($hs$) y aquellos con un título profesional ($coll$).\\


 
\begin{itemize}

   

    \item[a)] Expliquen qué tipo de relación captura $\tau$ entre $w_{lj}$ y $r_{lj}$. ¿Dado el contexto, será el estimador de MCO $\hat{\tau}$ consistente para $\tau$? Justifique su respuesta.\\
    \begin{solucion}
\end{solucion}
\end{itemize}   

Enfrentado a un problema de endogeneidad, el autor opta por una estrategia de variables instrumentales de tipo \textit{Shift-Share}. Card (2009) propone un instrumento candidato que toma la forma:

$$B_{lj}=\sum_c z_{lc,1980} \; g_{cj}$$

Donde, los ``shares'' están dados por 

$$z_{lc,1980}=\dfrac{N_{lc,1980}}{N_{c,1980}} \times \dfrac{1}{P_{l,2000}}$$

donde $N_{c,1980}$ es el número de inmigrantes provenientes de cada uno de los 38 países (indexados por $c$) que vivían en Estados Unidos en 1980, mientras que $N_{lc,1980}$ representa el total de inmigrantes provenientes del país $c$ pero que viven en la ciudad $l$. Finalmente, $P_{l,2000}$ es la población de la ciudad $l$ en el 2000. La idea de incluir este factor es expresar el número de migrates que llega a una ciudad $l$ como fracción de su población (simplemente es un ajuste de escala).\\

Por su parte, el ``shift'' está dado por $g_{cj}$ y es sencillamente el número total de personas que ingresaron \textit{a todo Estados Unidos} entre 1980 y el 2000 provenientes del país $c$ y con educación alcanzada $j$. Así las cosas, siguiendo la idea de Bartik, $B_{lj}$ es un potencial instrumento de $\ln(r_{lj})$.\\




\begin{itemize}

   

    \item[b)] Con base en la estrategia de Card:
    
    
     \begin{itemize}
         \item[i)] Discutan cuál de las dos aproximaciones (exogeneidad de los shares o exogeneidad de los shifts) consideran más apropiada según el contexto específico. Enfoquen su argumentación en los supuestos de identificación que se deben cumplir en cada caso, y expliquen por qué esos supuestos resultan plausibles (o no) dadas las características particulares del entorno de estudio. \\
        
         \begin{solucion}
\end{solucion}
           \item[ii)] Planteen la ecuación estructural, la primera etapa y la forma reducida asociada al problema.\\ 
         \begin{solucion}
\end{solucion}
     \end{itemize}
     
     

\end{itemize}   


El supuesto de relevancia establece que, condicional en $\boldsymbol{X}_l$, debe existir al menos un $k$ para el cual $z_{kl, 1980}$ tenga poder predictivo sobre $\ln(r_{lj})$ y que esta influencia sea tal que no se cancele al agregar los choques. Por su parte, exogeneidad estricta es la restricción de exclusión tradicional, pero esta recae sobre los ``shifts''. Esencialmente, lo que esta condición plantea es que las poblaciones de inmigrantes en 1980 provenientes de un país $c$, \textit{condicional en } $\boldsymbol{X}_l$, no deben estar correlacionados con el factor no observado de la brecha salarial observada en el 2000. Noten, sin embargo, que ello debe valer sólo para todos aquellos poblaciones donde hubo  efectivamente flujo de migrantes. En palabras más sencillas, lo que establecen ambas condiciones es que \textbf{la única manera por la cual tener distintas proporciones de inmigrantes de diferentes nacionalidades en 1980 pudo afectar la brecha salarial en los 2000 en una ciudad con características $\boldsymbol{X}_l$ es porque esa distribución particular atrajo más inmigrantes de las flujos de inmigración observados entre 1980 y el 2000.}

\begin{itemize}
    \item[c)] Utilizando el contexto y los datos provistos
    
    \begin{itemize}
     
    
    
    \item[i)] Estime $\tau$ por MCO y por MC2E tanto para la población con sólo título de secundaria como para aquellos con título de universidad. Muestre sus resultados en \textbf{una sóla tabla}. Incluya como controles el residual de los salarios para inmigrantes y nativos en 1980, el logaritmo de la población en 1980 y en 1990, así como la proporción de estudiantes en la universidad y la fracción que ocupaba la industria manufacturera en estos mismos años. Además, utilice como pesos para las regresiones la población de 1990. ¿Por qué es importante considerar estos pesos? \\
    
    Finalmente, para la estimación por MC2E, incluya el estadístico $F$. ¿Por qué es importante reportar e interpretar este estadístico cuando se hace variables instrumentales? \\
    
\begin{solucion}
\end{solucion}
    
    \item[ii)] Suponiendo que los supuestos de identificación se cumplen, interpreten los resultados.
    \begin{solucion}
\end{solucion}
    \item[iii)] Argumenten si, desde su punto de vista, los supuestos de identificación son válidos en el contexto del problema.
    \begin{solucion}
\end{solucion}

    
   
    
    \end{itemize}
    
\end{itemize}

Una ventaja importante de los instrumentos tipo Bartik es que se puede testear el supuesto de exogeneidad. Para lograrlo, se puede evaluar si los ``shifts'' (o aquellos más importantes) se correlacionan con otros factores que pueden explicar diferencias en la brecha salarial en el tiempo. Uno de los posibles ejercicios es el siguiente: \\

\begin{itemize}
    \item[d)] Utilizando los datos provistos:
    
    \begin{itemize}
        \item[i)] Completen la siguiente tabla:\\
    
    \end{itemize}
        
        \scalebox{0.6}{
\begin{tabular}{l*{8}{c}} \toprule
                &\multicolumn{1}{c}{Fracción de inm.}&\multicolumn{1}{c}{Fracción de inm.}&\multicolumn{1}{c}{Fracción de inm.}&\multicolumn{1}{c}{Fracción de inm.}&\multicolumn{1}{c}{Fracción de inm.}&\multicolumn{1}{c}{Fracción de inmig.}&\multicolumn{1}{c}{Bartik para}&\multicolumn{1}{c}{Bartik para}\\
                
                &\multicolumn{1}{c}{de México}&\multicolumn{1}{c}{de Filipinas}&\multicolumn{1}{c}{de El Salvador }&\multicolumn{1}{c}{de China}&\multicolumn{1}{c}{de Cuba}&\multicolumn{1}{c}{de Europa del Este}&\multicolumn{1}{c}{secundaria}&\multicolumn{1}{c}{universidad}\\
                
                &\multicolumn{1}{c}{1980}&\multicolumn{1}{c}{1980}&\multicolumn{1}{c}{1980 }&\multicolumn{1}{c}{1980}&\multicolumn{1}{c}{1980}&\multicolumn{1}{c}{+ Otros 1980}&\multicolumn{1}{c}{}&\multicolumn{1}{c}{}\\
\midrule
Log población 1980&   &  &  & &  & & & \\
                & & & & &  &  &  &  \\
Fracción univ. 1980&   & & & &  &  &  & \\
                &  &  &  &  & &  &  & \\
Salario res. nativo 1980& &  & & & &  &    & \\
                &  & & &  &  & &  &  \\
Salario res. inm. 1980&   & & & &  &  &   &  \\
                &   & & & & & &  &  \\
Fracción ind. manuf. 1980 &  & &  &  &  & &  &   \\
                &  &  &  &  &  &  &  &  \\
\midrule
$ R^2$          &   & & &  & &   &  &    \\
$N$               & &  & & & & & & \\
\bottomrule
\multicolumn{9}{c}{Errores estándar robustos a heterocedasticidad en parentésis}\\

\end{tabular}
} 
  

    \vspace{2mm}
    
    donde cada columna son los coeficientes y errores estándar resultantes de estimar un modelo de regresión múltiple de la variable indicada en la columna contra todas las variables explicativas señaladas en las filas.\\  

\vspace{3mm}      
  \textbf{Nota 1:} Por otros, se refiere a Australia, Chipre, Israel y Nueva Zelanda.\\
  
  \vspace{1mm}
  
  \textbf{Nota 2:} Para que los resultados sean legibles, reescale los coeficientes y errores estándar por 10,000,000.\\
  
  \vspace{1mm}
  
   \textbf{Nota 3:} Utilice la población en 1990 como pesos tal como se hizo en el inciso anterior. 
  \begin{solucion}
\end{solucion}
        

    
    \item[ii)] Interpreten los resultados a la luz de los supuestos de identificación. ¿La información encontrada refuta o apoya el cumplimiento de estos supuestos?  
    \begin{solucion}
\end{solucion}
    \end{itemize}
    


\end{document}
